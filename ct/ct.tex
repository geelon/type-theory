\documentclass[reqno]{amsart}
%----------Packages----------
\usepackage{amsmath,amssymb,amsthm,graphicx,tikz}
\usepackage{mathrsfs}
\usepackage[all,arc,cmtip,arrow,frame,matrix]{xy}
\usepackage{verbatim,float} 
\usepackage[top=1in,bottom=1in,left=1in,right=1in]{geometry}
\renewcommand{\baselinestretch}{1.15}
\usepackage[hidelinks]{hyperref}
\usepackage{mathtools}

\usepackage{setspace}
\setstretch{1.15}

\providecommand{\weakto}{\rightharpoonup}
\renewcommand{\vec}[1]{\mathbf{#1}}
\newcommand{\verteq}{\rotatebox{90}{$\,=$}}


%--Differentiation
\renewcommand{\d}[1]{\ensuremath{\,\mathrm{d}{#1}}}

\renewcommand{\tilde}[1]{\widetilde{#1}}
\providecommand{\tr}{\mathrm{tr}}
\renewcommand{\Re}{\mathrm{Re}}
\renewcommand{\Im}{\mathrm{Im}}

\let\oldto\to
\renewcommand{\to}{\arr}

\let\oldmod\mod
\renewcommand{\mod}[1]{\,\left(\mathrm{mod}\, #1 \right)}

\newcommand{\cl}[1]{\text{cl}\hspace{.05cm}(#1)}
\newcommand{\oline}[1]{\overline{#1}}

\providecommand{\sgn}[1]{\mathrm{sgn}\left( #1\right)}

\providecommand{\col}[1]{\begin{bmatrix} #1 \end{bmatrix}}

\let\propsubset\subset
\let\propsupset\supset
%\renewcommand{\subset}{\subseteq}
%\renewcommand{\supset}{\supseteq}
\providecommand{\anglebrac}[1]{\left\langle #1\right\rangle}

%------Proof Environment-------%
\renewcommand{\proof}[2][]{\noindent{\it Proof{#1}. } #2 \hfill $\diamondsuit$}
\newcommand{\df}[1]{\hbox{\hyperref[#1]{\textsc{def}#1}}}
\newcommand{\thm}[1]{\hbox{\hyperref[#1]{\textsc{thm}#1}}}
\newcommand{\cor}[1]{\hbox{\hyperref[#1]{\textsc{cor}#1}}}
\newcommand{\lem}[1]{\hbox{\hyperref[#1]{\textsc{lem}#1}}}
\newcommand{\ex}[1]{\hbox{\hyperref[#1]{\textsc{ex}#1}}}

\renewcommand{\and}{\textrm{ and }}

%---------Setminus-----------
\newcommand\rsetminus{\mathbin{\mathpalette\rsetminusaux\relax}}
\newcommand\rsetminusaux[2]{\mspace{-4mu}
  \raisebox{\rsmraise{#1}\depth}{\rotatebox[origin=c]{-20}{$#1\smallsetminus$}}
 \mspace{-4mu}
}
\newcommand\rsmraise[1]{%
  \ifx#1\displaystyle .8\else
    \ifx#1\textstyle .8\else
      \ifx#1\scriptstyle .6\else
        .45%
      \fi
    \fi
  \fi}

\renewcommand{\setminus}{\rsetminus}

\newcommand{\frakM}{\mathfrak{M}}
\newcommand{\frakE}{\mathfrak{E}}
%% bold math capitals
\newcommand{\bA}{\mathbf{A}}  \newcommand{\bB}{\mathbf{B}}  \newcommand{\bC}{\mathbf{C}}
\newcommand{\bD}{\mathbf{D}}  \newcommand{\bE}{\mathbf{E}}  \newcommand{\bF}{\mathbf{F}}
\newcommand{\bG}{\mathbf{G}}  \newcommand{\bH}{\mathbf{H}}  \newcommand{\bI}{\mathbf{I}}
\newcommand{\bJ}{\mathbf{J}}  \newcommand{\bK}{\mathbf{K}}  \newcommand{\bL}{\mathbf{L}}
\newcommand{\bM}{\mathbf{M}}  \newcommand{\bN}{\mathbf{N}}  \newcommand{\bO}{\mathbf{O}}
\newcommand{\bP}{\mathbf{P}}  \newcommand{\bQ}{\mathbf{Q}}  \newcommand{\bR}{\mathbf{R}}
\newcommand{\bS}{\mathbf{S}}  \newcommand{\bT}{\mathbf{T}}  \newcommand{\bU}{\mathbf{U}}
\newcommand{\bV}{\mathbf{V}}  \newcommand{\bW}{\mathbf{W}}  \newcommand{\bX}{\mathbf{X}}
\newcommand{\bY}{\mathbf{Y}}  \newcommand{\bZ}{\mathbf{Z}}

%% blackboard bold math capitals
\newcommand{\bbA}{\mathbb{A}}      \newcommand{\bbB}{\mathbb{B}}
\newcommand{\bbC}{\mathbb{C}}      \newcommand{\bbD}{\mathbb{D}}
\newcommand{\bbE}{\mathbb{E}}      \newcommand{\bbF}{\mathbb{F}}
\newcommand{\bbG}{\mathbb{G}}      \newcommand{\bbH}{\mathbb{H}}
\newcommand{\bbI}{\mathbb{I}}      \newcommand{\bbJ}{\mathbb{J}}
\newcommand{\bbK}{\mathbb{K}}      \newcommand{\bbL}{\mathbb{L}}
\newcommand{\bbM}{\mathbb{M}}      \newcommand{\bbN}{\mathbb{N}}
\newcommand{\bbO}{\mathbb{O}}      \newcommand{\bbP}{\mathbb{P}}
\newcommand{\bbQ}{\mathbb{Q}}      \newcommand{\bbR}{\mathbb{R}}
\newcommand{\bbS}{\mathbb{S}}      \newcommand{\bbT}{\mathbb{T}}
\newcommand{\bbU}{\mathbb{U}}      \newcommand{\bbV}{\mathbb{V}}
\newcommand{\bbW}{\mathbb{W}}      \newcommand{\bbX}{\mathbb{X}}
\newcommand{\bbY}{\mathbb{Y}}      \newcommand{\bbZ}{\mathbb{Z}}

%% script math capitals
\newcommand{\sA}{\mathscr{A}}  \newcommand{\sB}{\mathscr{B}} \newcommand{\sC}{\mathscr{C}}
\newcommand{\sD}{\mathscr{D}} \newcommand{\sE}{\mathscr{E}} \newcommand{\sF}{\mathscr{F}}
\newcommand{\sG}{\mathscr{G}} \newcommand{\sH}{\mathscr{H}} \newcommand{\sI}{\mathscr{I}}
\newcommand{\sJ}{\mathscr{J}} \newcommand{\sK}{\mathscr{K}} \newcommand{\sL}{\mathscr{L}}
\newcommand{\sM}{\mathscr{M}}\newcommand{\sN}{\mathscr{N}}\newcommand{\sO}{\mathscr{O}}
\newcommand{\sP}{\mathscr{P}} \newcommand{\sQ}{\mathscr{Q}} \newcommand{\sR}{\mathscr{R}}
\newcommand{\sS}{\mathscr{S}} \newcommand{\sT}{\mathscr{T}} \newcommand{\sU}{\mathscr{U}}
\newcommand{\sV}{\mathscr{V}} \newcommand{\sW}{\mathscr{W}} \newcommand{\sX}{\mathscr{X}}
\newcommand{\sY}{\mathscr{Y}} \newcommand{\sZ}{\mathscr{Z}}

\renewcommand{\emptyset}{\varnothing}

\providecommand{\abs}[1]{\left\lvert #1 \right\rvert}
\providecommand{\norm}[1]{\left\lVert #1 \right\rVert}

%\providecommand{\ar}{\rightarrow}
\providecommand{\arr}{\longrightarrow}
\providecommand{\sarr}{\rightarrow}
\newcommand*\surjects{\ensuremath{\relbar\joinrel\twoheadrightarrow}}

\renewcommand{\_}[1]{\underline{ #1 }}
\DeclareMathOperator{\ext}{ext}

\newcounter{axiomsnum}
\setcounter{axiomsnum}{-1}
%----------Theorems----------

\newtheorem{theorem}{Theorem}[section]
\newtheorem{proposition}[theorem]{Proposition}
\newtheorem{lemma}[theorem]{Lemma}
\newtheorem{fact}[theorem]{Fact}
\newtheorem{corollary}[theorem]{Corollary}
\theoremstyle{definition}
\newtheorem{example}[theorem]{Example}
\newtheorem{aside}{Aside}
\newtheorem{question}{Question}
\newtheorem{axiom}[axiomsnum]{Axiom}
\newtheorem{axioms}[axiomsnum]{Axioms}
\newtheorem{definition}[theorem]{Definition}
\newtheorem{nondefinition}[theorem]{Non-Definition}
\newtheorem{exercise}[theorem]{Exercise}
\newtheorem{examples}[theorem]{Examples}
\newtheorem{remark}[theorem]{Remark}
\newtheorem{warning}[theorem]{Warning}
%\numberwithin{equation}{section}


\theoremstyle{plain}
\newtheorem{postulate}{Postulate}


\newcommand{\problem}[2]{\noindent{\textbf{#1.} \emph{#2}}}

\providecommand{\Hom}{\mathrm{Hom}}
\providecommand{\End}{\mathrm{End}}
\providecommand{\id}{\mathrm{id}}

%%% Indented environment
%%% \indEnv{lemma}{lemma statement}{proof}
\providecommand{\indEnv}[3]{
\begin{quote}\vspace{-20pt}\begin{#1} #2 \end{#1} #3\end{quote}}


\providecommand{\dom}{\mathrm{dom}}
\providecommand{\cod}{\mathrm{cod}}
\providecommand{\Sets}{\mathbf{Sets}}
\providecommand{\Pos}{\mathbf{Pos}}
\providecommand{\One}{\mathbf{1}}
\providecommand{\Cat}{\mathbf{Cat}}
\providecommand{\Rel}{\mathbf{Rel}}
\providecommand{\op}[1]{{#1^{\mathrm{op}}}}
\providecommand{\Monoids}{\mathbf{Monoids}}

\title{Notes from Awodey's Category Theory}
\author{Aaron Geelon So}
\date{}

\begin{document}
\maketitle

\section{Basics}
A \textbf{category} consists of \emph{objects} and \emph{arrows}. For each arrow $f$ there are objects:
\[\dom(f),\quad \cod(f)\]
called the \emph{domain} and \emph{codomain} of $f$. We write $f:A \to B$. There is a composition of arrows $g \circ f$ when $\cod(f) = \dom(g)$. And, for each object $A$, there is an arrow $1_A: A\to A$ called the \emph{identity arrow} of $A$. Composition must be associative, and
\[f \circ 1_A = f = 1_B \circ f\]
for all $f: A \to B$.

Examples of categories are $\Sets$, the category of sets and functions. There is also a category of all finite sets and functions between them. Or, a category of finite sets as objects and injective functions as arrows. Other examples are \emph{structured sets}, or sets with some ``structure'' and functions that ``preserve'' it. For example, groups and group homomorphisms, vector spaces and linear maps, topological spaces and continuous maps, differentiable manifolds and smooth maps, posets and monotone functions, and so on. Denote the category of posets by $\Pos$. As another example, if $X$ is a set, then $\mathbf{Dis}(X)$ is a \emph{discrete} category where the only arrows are the identity arrows.

These are examples of \emph{concrete categories}, where the the objects are sets and the arrows are functions. However, there are categories whose objects are not sets or arrows that are not functions. For example, the category $\mathbf{Rel}$ of relations on sets has as arrows $f: A \to B$ subsets $f \subset A\times B$. The identity is $1_A = \{(a,a) : a \in A\}$ and the composite of $f: A\to B$ and $g: B \to C$ is
\[g \circ f = \{(a,c): (a,b) \in f \textrm{ and } (b,c) \in g\}.\]
The objects of a category do not need to be sets either. The category $\One$ is just:
\[*\]
a single object with its identity arrow. The category $\mathbf{2}$ is:
\[* \longrightarrow *\]
two objects with the required identity arrows, and one arrow between. The category $\mathbf{0}$ has no objects and no arrows. 

A `homomorphism of categories' is called a \textbf{functor}. Specifically, a functor $F: \mathbf{C} \to \mathbf{D}$ between categories $\mathbf{C}$ and $\mathbf{D}$ is a mapping of objects to objects and arrows to arrows where 
\begin{enumerate}
\item[1.] $F(f: A \to B) = F(f): F(A) \to F(B)$,
\item[2.] $F(g \circ f) = F(g) \circ F(f)$, and
\item[3.] $F(1_A) = 1_{F(A)}$.
\end{enumerate}
Since every category has an identity functor $1_\mathbf{C}$, the category of all categories $\Cat$ is another example of a category, where functors are the arrows.

\problem{Problem 1}{The objects of $\Rel$ are sets, and arrow $f: A \to B$ is a relation for $A$ to $B$, that is, $f \subseteq A \times B$. The identity relation $\{\langle a,a\rangle : a \in A\}$ is the identity arrow on a set $A$. Composition in $\Rel$ is to be given by 
\[g \circ f = \{ \langle a,c\rangle \in A \times C : \exists b\left(\langle a,b\rangle \in f \ \mathrm{and}\ \langle b,c\rangle \in g\right)\}\]
for $f \subseteq A\times B$ and $g \subseteq B \times C$. Show that $\Rel$ is a category.}

To show that $\Rel$ is a category, we just need to check composition associativity and composition with identity. Let $f : A \to B$, $g: B \to C$ and $h: C \to D$. Then,
\begin{align*}
h \circ (g \circ f) 
&= \{\langle a,d\rangle :\exists c\left(\exists b\left(\langle a,b\rangle \in f \ \mathrm{and} \ \langle b,c\rangle \in g\right) \ \mathrm{and} \ \langle c,d\rangle \in h\right)\}\\
&=  \{\langle a,d\rangle :\exists b\left( \langle a,b\rangle \in f \ \mathrm{and} \ \exists c\left(\langle b,c\rangle \in g \ \mathrm{and} \ \langle c,d\rangle \in h\right)\right)\} = (h \circ g) \circ f.
\end{align*}
In addition, $\langle a,b\rangle \in 1_B \circ f$ implies that there is some $c$ such that $\langle c,b \rangle \in 1_B$ and $\langle a,c\rangle \in f$. In particular, $c = b$, so that $\langle a,b\rangle \in f$. Thus,
\[f \subseteq 1_B \circ f.\]
Now, if $\langle a,b\rangle \in f$, and since $\langle b,b\rangle \in 1_B$, this implies that $\langle a , b\rangle \in 1_B \circ f$. Hence, $1_B \circ f = f$. The analogous argument shows that $f \circ 1_A = f$. This confirms that $\Rel$ is a category.\\

\problem{Problem 2}{Consider the following isomorphisms of categories and determine which hold.
\begin{enumerate}
\item[(a)] $\Rel \cong \op{\Rel}$.
\item[(b)] $\Sets \cong \op{\Sets}$.
\item[(c)] For a fixed set $X$ with powerset $P(X)$, as poset categories $P(X) \cong \op{P(X)}$, where the arrows in $P(X)$ are subset inclusions.
\end{enumerate}
}

\subsubsection*{Part A} The categories $\Rel$ and $\Rel^{\mathrm{op}}$ are isomorphic. Define $F: \Rel \to \op{\Rel}$ by
\[F(A) = \overline{A} \textrm{ and } F(f: A \to B) = F(f): \overline{A} \to \overline{B},\]
where $F(f) = \overline{f^{-1}}$. We see that $F$ is a functor since
\[F(g\circ f) = \overline{(g \circ f)^{-1}} = \overline{ f^{-1} \circ g^{-1}} = \overline{g^{-1}} \circ \overline{f^{-1}} = F(g) \circ F(f),\]
and that
\[F(1_A) = \overline{1_A^{-1}} = \overline{1_A} = 1_{\overline{A}} = 1_{F(A)}.\]
Similarly, we can construct $G: \op{\Rel} \to \Rel$ by
\[G(\overline{A}) = A \textrm{ and } G(\overline{g}:\overline{A} \to \overline{B}) = G(\overline{g}): A \to B,\]
where $G(\overline{g}) = g^{-1}$. Now, $G$ is a functor since
\[G(\overline{g} \circ \overline{f}) = f^{-1} \circ g^{-1} = \overline{g^{-1}} \circ \overline{f^{-1}} = G(\overline{g}) \circ G(\overline{f}),\]
and that
\[G(1_{\overline{A}}) = G(\overline{1_A}) = 1_A^{-1} = 1_A = 1_{G(A)}.\]
Finally, $(G \circ F) (f) = G(\overline{f^{-1}}) = f$ and $(G \circ F)(A) = G(A) = A$, so $G \circ F = 1_{\Rel}$. Similarly, we find that $F \circ G = 1_\op{\Rel}$. Therefore, $\Rel$ and $\op\Rel$ are isomorphic.

\subsubsection*{Part B} The categories $\Sets$ and $\op\Sets$ are not isomorphic. Consider the two sets $\emptyset$ and $\{*\}$. Define $f: \emptyset \to \{*\}$. In $\op\Sets$, the corresponding arrow is $\overline{f} : \{*\} \to \emptyset$. However, $\overline{f}$ is clearly not a function since $f(*)$ is an element of $\emptyset$. There is no isomorphism of $\Sets$ and $\op\Sets$.

\subsubsection*{Part C} The categories $P(X)$ and $\op{P(X)}$ are isomorphic. Define the functor $F: P(X) \to \op{P(X)}$ by
\[F(A) = \overline{A^c}\textrm{ and } F(f: A \subseteq B) = F(f): \overline{A^c} \supseteq\overline{B^c},\]
where $A \subseteq X$. To see that $F$ is a functor, let $f: A \subseteq B$ and $g: B \subseteq C$. Then, we know that $A^c \supseteq B^c$ and $B^c \supseteq C^c$. In particular, this implies that $F(f)$ and $F(g)$ are witnesses, and that
\[F(g\circ f) = F(g) \circ F(f).\]
A similar argument for the analogous functor $G: \op{P(x)} \to P(X)$ can be done. Then, we have
\[(G \circ F)(A) = G(A^c) = (A^c)^c = A,\]
and that
\[G(\circ F)(A \subseteq B) = G( A^c \supseteq B^c) = (A^c)^c \subseteq (B^c)^c = A \subseteq B.\]
Thus, $G \circ F = 1_{P(X)}$ and likewise, $F\circ G = 1_{\op{P(X)}}$.\\

\problem{Problem 3}{\begin{enumerate}
\item[(a)] Show that in $\Sets$, the isomorphisms are exactly the bijections.
\item[(b)] Show that in $\Monoids$, the isomorphisms are exactly the bijective homomorphisms.
\item[(c)] Show that in $\Pos$, the isomorphisms are \emph{not} the same as the bijective homomorphisms.
\end{enumerate}}

\subsubsection*{Part A} Let $f: A \to B$ be an isomorphism of sets; that is, there is some arrow $g: B \to C$ such that $g \circ f = 1_A$ and $f\circ g = 1_B$. If $f(x) = f(y)$, then $x = g(f(x)) = g(f(y)) = y$, so that $f$ is injective. And for any $y \in B$, $g(f(g(y))) = y$, so that $f$ is surjective. Thus, every set isomorphism is a bijection. And if $f$ is a bijection, then it has an inverse $f^{-1}$, where $f^{-1} \circ f = 1_A$ and $f \circ f^{-1} = 1_B$. Thus, every bijection is a set isomorphism.

\subsubsection*{Part B} Let $f : A \to B$ be an isomorphism of monoids. Then, there exists a homomorphism $g: B \to A$ such that $g \circ f = 1_A
$ and $f \circ g = 1_B$. This implies that $f$ has an inverse, so that $f$ is a bijective homomorphism. Now, let $f$ be a bijective homomorphism. Let $g$ be the inverse of $f$. We just need to show that $g$ is a homomorphism of monoids. Since homomorphisms map the unit of monoids to another unit, this implies that $g(u_B) = u_A$, where $u_A$ and $u_B$ are the units of the monoids $A$ and $B$. Further, let $m,n \in B$, where $m = f(a)$ and $n = f(b)$. Then
\[g(m \cdot_B n) = g(f(a) \cdot_B f(b)) = g(f(a \cdot_A b)) = a \cdot_A b = g(m) \cdot_B g(n).\]
This shows that $g$ is a homomorphism, and so $f$ is an isomorphism.

\subsubsection*{Part C} Consider the posets $A$ and $B$, where
\[\mathrm{Objects}(A) = \{a,b,c\} \textrm{ and } \mathrm{Objects}(B) = \{x,y,z\},\] 
and the arrows besides the identity arrows are
\[\mathrm{Morphisms}(A) = \{a \to b, a \to c\} \textrm{ and } \mathrm{Morphisms}(B) = \{ x \to y, x \to z, y\to z\}.\]
Let $F:A \to B$ be the functor uniquely defined by $F(a) = x$, $F(b) = y$, and $F(c) = z$. This functor is a bijective homomorphism. However, it is not an isomorphism of $A$ and $B$. In fact, $A$ and $B$ cannot be isomorphic because the number of morphisms differ.
\end{document}

%%% Local Variables:
%%% mode: latex
%%% TeX-master: t
%%% End:
